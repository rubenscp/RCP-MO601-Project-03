
\subsection{Simulador Multi-core Sniper}

Sniper: Instalar e executar no seu computador. Executar 3 programas pequenos (< 1 milhão de instruções cada). Entrega: sequência de comandos executados, slowdown de simulação (tempo do programa executado no simulador dividido pelo tempo do programa executando nativamente), colete e apresente algumas métricas de desempenho coletadas pelo simulador. \\

Sniper é um simulador do processador x86 para computação paralela, alta velocidade e acurácia.

\begin{itemize}
    \item Site oficial: https://snipersim.org//w/The\_Sniper\_Multi-Core\_Simulator.
    \item Instalação: foram seguidas as instruções disponibilizadas na página do produto.
\end{itemize}

Foi instalado também o Pin conforme instruções no site do Sniper. \\


Verificar o resultado produzido pelo pin:

\begin{itemize}
    \item cd source/tools/SimpleExamples
    \item make obj-intel64/opcodemix.so
    \item ../../../pin -t obj-intel64/opcodemix.so -- /bin/ls
    \item Acessar a pasta: benchmark/sniper/sniper-7.4/pin-kit/pin-3.27-98718-gbeaa5d51e-gcc-linux
    \item Acessar a pasta: source/tools/SimpleExamples
    \item Verificar os arquivos com extensão $out$.
\end{itemize}


Programas a serem testados:

\begin{enumerate}
    \item opcodemix.cpp
    \item coco.cpp 
    \item trace.cpp
    \item catmix.cpp 
\end{enumerate}


\textbf{Entrega}

\begin{enumerate}
    \item Sequência de comandos executados.
    \item Slowdown de simulação (tempo do programa executado no simulador dividido pelo tempo do programa executando nativamente).
    \item Apresentar algumas métricas de desempenho coletadas pelo simulador.
\end{enumerate}
