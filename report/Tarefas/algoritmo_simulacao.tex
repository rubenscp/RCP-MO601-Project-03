\section{Algoritmo de Simulação}

O algoritmo do simulador do processador RISC-V RV32IM é apresentado em \textit{algorithm \ref{alg:simulacao}}.

\BlankLine

\begin{algorithm}[ht]
    \caption{Algoritmo do simulador do processador RISC-V RV32IM.}
    \label{alg:simulacao}
    
    % defining key words to use in the text
    \SetKwInput{KwEntrada}{Entrada}
    \SetKwInOut{KwSaida}{Saída}
    \SetKwFunction{KwExecutarPrograma}{ExecutarPrograma}

    \SetAlgoLined
    
    \BlankLine

    \KwEntrada {\textit{programa assembler (exemplo: 000.main.asm)}}

    \BlankLine

    \KwSaida {\textit{arquivo de saída (log) do programa simulado (exemplo: 000.main.log)}}

    \BlankLine

    \KwExecutarPrograma{programa assembler, PC}
    \Begin{
        
        \While{não alcançar a instrução \textit{ebreak}}
        {
            \BlankLine

            ler instrução indicada pelo PC

            \BlankLine

            decodificar instrução conforme o tipo específico (R/I/S/B/U/J)

            \BlankLine

            executar a instrução de acordo com sua implementação 

            \BlankLine

            gerar linha de log de saída da instrução executada
        }        
    }
            
    \BlankLine

    \BlankLine

    \CommentSty{/* Programa principal que gerencia a simulação do processador RISC-V RV32IM na execução de programas. */}

    \BlankLine

    \ForEach{programa assembler}
    {    
        inicializar processador RISC-V criando memória RAM, registradores e PC

        \BlankLine

        ler programa assembler

        \BlankLine

        armazenar programa lido na memória RAM

        \BlankLine

        inicializar contador do programa (PC) com o endereço de memória da primeira instrução que será executada

        \BlankLine
        
        \KwExecutarPrograma{programa assembler, PC}

        \BlankLine

        gravar arquivo com o log das instruções executadas

        \BlankLine

        gravar estatística da simulação

    } 

\end{algorithm}