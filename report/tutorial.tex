

%Para tabela use:
\begin{table}[H]
	\centering
	\caption{Exemplo de tabela}
	\label{tab:nivelz}
	\begin{tabular}{l l}
		\hline
		Nível & valor \\ 
		\hline
		68,3 \%      &   1,00\\
		95 \%        &   1,96\\
		99 \%$^a$    &   1,99$^b$\\
		\hline
		\footnotesize{$^a$ legenda} \\
		\footnotesize{$^b$ legenda}\\
	\end{tabular}
\end{table}



%Para figura use
\begin{figure}[H]
    \centering
    \caption {Exemplo de figura}
    \includegraphics[scale=0.6]{Imagens/LOGO-EQ.jpg}
    \label{fig:logo}
\end{figure}


\begin{equation}
     R = \frac{V}{i} 
     \textrm{\hspace{0.5cm}, onde}
    \begin{cases}
        \text{R (constante)} \\
        \text{i é corrente (A)}
    \end{cases}
    \label{eq:leiohm}   
\end{equation}


%química
\ce{Na2SO4  ->[H2O] Na+ + SO4^2-}

\ce{(2Na+,SO4^2- ) + (Ba^2+, 2Cl- ) -> BaSO4 v + 2NaCl}



%Para citar tabela ou figura use
\ref{fig:LN}



%Para citar referencia use
\cite{Cataluna2012}



%Para nova seção ou subseção
\section{Example1}
\subsection{Example2}
\subsubsection{Example3}


%pular linha
\vphantom{}
\vspace{69mm}


%marcadores
\fontsize{15}{\baselineskip}\selectfont
\text{Alocação dos trocadores }



\begin{itemize}
   \item Analise do mercado
\end{itemize}





% Para nomenclatura
\nomenclature[A, 01]{$x$}{Speed of light in a vacuum inertial frame}

\nomenclature[A, 01]{$c$}{speed of light
    \nomunit{$[ m \cdot s]$}}

\nomenclature[A, 01]{$g$}{Gravitational Constant 
    \nomunit{$[m \cdot s^{-2}]$}}




